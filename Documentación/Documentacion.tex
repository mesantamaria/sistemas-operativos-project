% Plantilla para documentos LaTeX para enunciados
% Por Pedro Pablo Aste Kompen - ppaste@uc.cl
% Licencia Creative Commons BY-NC-SA 3.0
% http://creativecommons.org/licenses/by-nc-sa/3.0/

\documentclass[12pt]{article}

% Margen de 1 pulgada por lado
\usepackage{fullpage}
% Incluye gráficas
\usepackage{graphicx}
% Packages para matemáticas, por la American Mathematical Society
\usepackage{amssymb}
\usepackage{amsmath}
% Desactivar hyphenation
\usepackage[none]{hyphenat}
% Saltar entre párrafos - sin sangrías
\usepackage{parskip}
% Español y UTF-8
\usepackage[spanish]{babel}
\usepackage[utf8]{inputenc}
% Links en el documento
\usepackage{hyperref}
\usepackage{fancyhdr}
\setlength{\headheight}{15.2pt}
\setlength{\headsep}{5pt}
\pagestyle{fancy}
\usepackage{cite}
\usepackage{listings}


\newcommand{\N}{\mathbb{N}}
\newcommand{\Exp}[1]{\mathcal{E}_{#1}}
\newcommand{\List}[1]{\mathcal{L}_{#1}}
\newcommand{\EN}{\Exp{\N}}
\newcommand{\LN}{\List{\N}}

\newcommand{\comment}[1]{}
\newcommand{\lb}{\\~\\}
\newcommand{\eop}{_{\square}}
\newcommand{\hsig}{\hat{\sigma}}
\newcommand{\ra}{\rightarrow}
\newcommand{\lra}{\leftrightarrow}

\def\code#1{\texttt{#1}}
\def\ccode#1{\begin{center}\texttt{#1}\end{center}}

\begin{document}
\thispagestyle{empty}
% Membrete
% PUC-ING-DCC-IIC1103
\begin{minipage}{2.3cm}
\includegraphics[width=2cm]{img/logo.pdf}
\vspace{0.5cm} % Altura de la corona del logo, así el texto queda alineado verticalmente con el círculo del logo.
\end{minipage}
\begin{minipage}{\linewidth}
\textsc{\raggedright \footnotesize
Pontificia Universidad Católica de Chile \\
Departamento de Ciencia de la Computación \\
IIC2333 - Sistemas operativo y Redes \\}
\end{minipage}


% Titulo
\begin{center}
\vspace{-0.5cm}
{\huge\bf Documentación SIGKILLERS}\\
\vspace{0.2cm}
\today\\
\vspace{0.2cm}
\footnotesize{Primer semestre 2019}\\
\vspace{0.2cm}
\end{center}

\section*{Funciones}

\subsection*{Generales}
\begin{itemize}
	\item \code{void cr\_mount(char* diskname)}
	\item \code{void cr\_bitmap()}
	\item \code{int cr\_exists(char* path)}
	\item \code{void cr\_ls(char* path)}
	\item \code{int cr\_mkdir(char* foldername)}
\end{itemize}

\subsection*{Archivos}
\begin{itemize}
	\item \code{crFILE* cr\_open(char* path, char mode)}
	\item \code{int cr\_read(crFILE* file\_desc, void* buffer, int nbytes)}
	\item \code{int cr\_write(crFILE* file\_desc, void* buffer, int nbytes)}
	\item \code{int cr\_close(crFILE* file\_desc)}
	
	Se preocupa de liberar los recursos pedidos por \code{cr\_open}.
	
	\item \code{int cr\_rm(char* path)}
	
	Función que se encarga de borrar el archivo referenciado por la ruta \code{path}.

	Solo borra el archivo si no quedan más \code{hardlinks} que apunten a el. En caso contrario, solo se borra la referencia en el \code{path} correspondiente.
	
	Si todo sale bien, se retorna \code{0}. En caso de que el \code{path} indicado no exista, se imprimirá un mensaje en consola indicando esto y se retornará \code{0}
	\item \code{int cr\_hardlink(char* orig, char* dest)}
	\item \code{int cr\_unload(char* orig, char* dest)}
	\item \code{int cr\_load(char* orig)}
\end{itemize}


\end{document}
